\documentclass{article}
\usepackage{minted}
\usepackage[dvipsnames,svgnames]{xcolor}
\usepackage{hyperref}
\usepackage{cprotect}

\begin{document}

\title{Syntax Highlighting with minted Package}

\maketitle

\tableofcontents

\listoflistings   % List source code

\section{Basic Usage}

\subsection{Simple examples}

\begin{minted}{c}
/* Hello, world */
int main{int argc, char **argv} {
  printf("Hello, world\n");
  return 0;
}
\end{minted}

This is a single line code: \mint{python}|import pyqt5|

This is a inline code: \mintinline{python}|import pyqt5|

Input form a file:

\inputminted[]{c}{../code/helloworld.c}

\subsection{Supported Languages and Styles}

List of the currently supported languages, use the command: \mint{bash}|pygmentize -L lexers|

To get a list of all available stylesheets, see the online demo at the 
\href{https://pygments.org/demo/}{Pygments website} or 
\href{https://thepythonguru.com/tools/pygments-demo/}{Syntax Highlighter}
or execute the following command on the command line: \mint{bash}|pygmentize -L styles|

\cprotect\section{\verb|listing| Enviroment}

\begin{listing}[H]
  \inputminted[]{c}{../code/helloworld.c}
  \caption{A minimal C program}
  \label{lst:helloworld}
\end{listing}

Listing \ref{lst:helloworld} is a minimal C program.

\section{Styles}

\subsection{default}
\usemintedstyle{default}
\inputminted[bgcolor=AntiqueWhite]{c}{../code/helloworld.c}

\subsection{emacs}
\usemintedstyle{emacs}
\inputminted[bgcolor=AntiqueWhite]{c}{../code/helloworld.c}

\subsection{friendly}
\usemintedstyle{friendly}
\inputminted[bgcolor=AntiqueWhite]{c}{../code/helloworld.c}

\subsection{colorful}
\usemintedstyle{colorful}
\inputminted[bgcolor=AntiqueWhite]{c}{../code/helloworld.c}

\subsection{autumn}
\usemintedstyle{autumn}
\inputminted[bgcolor=AntiqueWhite]{c}{../code/helloworld.c}

\subsection{murphy}
\usemintedstyle{murphy}
\inputminted[bgcolor=AntiqueWhite]{c}{../code/helloworld.c}

\subsection{manni}
\usemintedstyle{manni}
\inputminted[bgcolor=AntiqueWhite]{c}{../code/helloworld.c}

\subsection{monokai}
\usemintedstyle{monokai}
\inputminted[bgcolor=AntiqueWhite]{c}{../code/helloworld.c}

\subsection{pastie}
\usemintedstyle{pastie}
\inputminted[bgcolor=AntiqueWhite]{c}{../code/helloworld.c}

\subsection{sas}
\usemintedstyle{sas}
\inputminted[bgcolor=AntiqueWhite]{c}{../code/helloworld.c}

\subsection{solarized-dark}
\usemintedstyle{solarized-dark}
\inputminted[bgcolor=AntiqueWhite]{c}{../code/helloworld.c}

\subsection{xcode}
\usemintedstyle{xcode}
\inputminted[bgcolor=AntiqueWhite]{c}{../code/helloworld.c}

\subsection{tango}
\usemintedstyle{tango}
\inputminted[bgcolor=AntiqueWhite]{c}{../code/helloworld.c}

\section{Troubleshooting}

\subsection{mintline and bgcolor}

Breaklines in mintline does not work when bgcolor is setted. 
See 
\href{https://github.com/gpoore/minted/issues/194}{Github issue 194: breaklines doesn't work with mintinline when other options are set}
and 
\href{https://tex.stackexchange.com/questions/419934/breaklines-doesnt-work-with-mintinline}{Stackoverflow: breaklines doesn't work with mintinline}

The documentation mentions this as a limitation of bgcolor. The standard ways to put a background behind inline text don't work with line wrapping.

\subsection{Page break with large source file}

See
\href{https://tex.stackexchange.com/questions/368864/pagebreak-for-minted-in-figure}{Stackoverflow: Pagebreak for minted in figure}
and
\href{https://tex.stackexchange.com/questions/12428/code-spanning-over-two-pages-with-minted-inside-listing-with-caption/53540#53540}{Code spanning over two pages with minted, inside listing with caption}

\subsection{How to disable red boxes around non usual characters?}

See
\href{https://tex.stackexchange.com/questions/343494/minted-red-box-around-greek-characters}{Minted red box around greek characters}
and 
\href{https://tex.stackexchange.com/questions/424421/code-validation-in-minted-package-how-to-disable-it}{Code validation in minted package? How to disable it?}

Use one of the following styles:
xcode, igor, rrt.
\end{document}